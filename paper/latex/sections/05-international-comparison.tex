% ============================================================================
% Section 5: International Comparative Analysis
% ============================================================================

\section{国際比較分析}
\label{sec:international-comparison}

本節では、第\ref{subsec:six-axes}節で提示した6軸比較フレームワーク——非党派性・非企業性・オープンソース度・制度的接合性・参加の包摂性・エージェントレディ度——を用いて、台湾・英国・米国・欧州(スペイン・フランス・ドイツ・アイスランド)・日本の5地域における\politech{}の展開を比較分析する。

% ----------------------------------------------------------------------------
\subsection{台湾——g0v/vTaiwanモデル}
\label{subsec:taiwan}

台湾は、世界で最も先進的な\politech{}エコシステムを構築している地域の一つであり、市民ハッカーコミュニティの自発的発展と政治制度への制度的接合が高度に統合された稀有な事例である。

\subsubsection{g0v(零時政府)の発展}

g0v(gov-zero、零時政府)は2012年12月に発足した台湾の\civictech{}/\politech{}コミュニティであり、「Fork the Government(政府をフォークせよ)」をスローガンに掲げる\autocite{tang2024plurality}。その名称は、政府(gov)のドメインの「o」を「0」に置き換えるという象徴的行為に由来し、既存の政府サービスに対するオープンソースの代替物を市民が自ら構築するという理念を表現している。

g0vは隔月でハッカソンを開催し、設立以来200回以上の大小のハッカソンを実施してきた。このハッカソンは、単なる技術イベントではなく、市民・技術者・ジャーナリスト・公務員が混在する「交差空間(boundary space)」として機能し、\civictech{}と\politech{}の連続的なイノベーションを生み出す基盤となっている。

g0vから生まれた主要プロジェクトは以下の通りである。
\begin{itemize}[nosep]
  \item \textbf{政治献金数位化(Campaign Finance Digitization)}——紙の政治献金報告書をクラウドソーシングでデジタル化するプロジェクト。2014年の公開後24時間以内に310,833筆の報告書がデジタル化された。
  \item \textbf{立院影城(LY.g0v.tw)}——立法院(国会)の議事録・投票記録の可視化プラットフォーム。
  \item \textbf{預算視覚化(Budget Visualization)}——政府予算のインタラクティブな可視化。
  \item \textbf{Cofacts}——LINEメッセージの真偽検証のための協働型ファクトチェックボット。
\end{itemize}

\subsubsection{vTaiwanモデル}

vTaiwanは2014年に発足した参加型政策形成プラットフォームであり、g0vコミュニティと行政院(内閣)の協働によって運営されてきた。その設計の核心は、オンライン意見集約とオフライン熟議の統合にある。

vTaiwanのプロセスは以下の4段階から構成される\autocite{tang2024plurality}。
\begin{enumerate}[nosep]
  \item \textbf{提案段階}——政策課題が設定され、背景情報が公開される。
  \item \textbf{意見段階}——Pol.isを用いた大規模意見集約が行われる。参加者は短文の意見に対して「同意」「不同意」「パス」で回答し、その回答パターンが主成分分析によってクラスタリングされる。
  \item \textbf{熟議段階}——Pol.isで可視化された意見分布をもとに、ステークホルダーが対面で熟議を行う。この段階では、クラスタ間の「合意点(consensus statements)」の発見に焦点が当てられる。
  \item \textbf{立法段階}——熟議の結果が法案に反映され、行政院を通じて立法化される。
\end{enumerate}

vTaiwanは2015年から2020年の間に26の政策課題を扱い、UberX規制、オンラインアルコール販売規制、遠隔医療規制、フィンテック規制などの政策を立法化した。26課題のうち約80\%が何らかの形で法制化に至ったとされる。

\subsubsection{Audrey Tangと制度的接合}

vTaiwanモデルの成功を語る上で、唐鳳(Audrey Tang)の存在は不可欠である。Tangはg0vコミュニティの中核メンバーとして活動した後、2014年のひまわり学生運動(Sunflower Movement)——中台サービス貿易協定に反対する学生が立法院を占拠した事件——を経て、2016年に蔡英文政権のデジタル大臣(政務委員)に35歳で任命された。

Tangの任命は、市民ハッカーが政治制度の内部に参入するという、\civictech{}から\politech{}への転換を象徴的に示す出来事であった。Tangは「ラディカルな透明性(radical transparency)」を掲げ、自身のすべての会議記録を公開し、g0vコミュニティとの連携を制度化した。

さらに、Join.gov.tw(公共政策網路参与平台)は2015年に立ち上げられた政府公式の市民参加プラットフォームであり、1,200万人以上(台湾の人口の約半数)が利用している。5,000筆以上の署名を集めた提案には政府が60日以内に回答する義務があり、制度的接合性の高さを示している。

大統領ハッカソン(Presidential Hackathon)は2018年から毎年開催され、市民・公務員・技術者の混成チームが政策課題に取り組む。優秀プロジェクトには大統領自らがトロフィーを授与し、行政機関による実装が約束される。

\subsubsection{6軸評価}

台湾のg0v/vTaiwanモデルの6軸評価は以下の通りである。

\begin{description}[style=nextline,leftmargin=3em]
  \item[非党派性: 4/5] g0vコミュニティは明示的に非党派であり、「自分でやれ(Nobody)」の原則を掲げる。ただし、vTaiwanは民進党政権下で発展し、国民党政権への移行後は活動が縮小した。制度的接合が特定政権に依存する点は非党派性の限界である。
  \item[非企業性: 5/5] g0v・vTaiwanは営利企業からの組織的独立性を維持している。Pol.isもオープンソースであり、企業依存度は極めて低い。
  \item[オープンソース度: 5/5] g0vのプロジェクトはすべてGitHub上で公開されている。Pol.isもオープンソースであり、コードの検証可能性は完全に確保されている。
  \item[制度的接合性: 4/5] vTaiwanは行政院との連携により立法化までの経路が制度化されていた。Join.gov.twは法令に基づく回答義務を有する。ただし、vTaiwanの制度的位置づけは法定されておらず、政権交代に対する脆弱性がある。
  \item[参加の包摂性: 3/5] Pol.isは低い参加障壁を実現しているが、デジタルリテラシーの格差は残る。vTaiwanの参加者は都市部の高学歴層に偏る傾向が報告されている。
  \item[エージェントレディ度: 2/5] Pol.isはAPIを提供しているが、エージェントの体系的な参入を前提とした設計は行われていない。Join.gov.twの構造化されたAPIも限定的である。
\end{description}

% ----------------------------------------------------------------------------
\subsection{英国——mySocietyとオープンデータ}
\label{subsec:uk}

英国は、情報公開法(Freedom of Information Act 2000)に基づくオープンデータ文化と、mySocietyに代表される\civictech{}の長い伝統を有する。その特徴は、既存の民主制度への「接木(grafting)」型のアプローチにある。

\subsubsection{mySocietyエコシステム}

mySocietyは2003年にTom Steinbergによって設立された非営利団体であり、英国の\civictech{}を国際的に牽引してきた\autocite{oecd2025civic}。同団体の設計哲学は「ユーザーニーズに基づく公共サービスのデジタル化」であり、技術的洗練度と市民のニーズへの応答性を両立させている。

主要プラットフォームの影響は以下の通りである。
\begin{itemize}[nosep]
  \item \textbf{WhatDoTheyKnow}——英国最大の情報公開請求プラットフォーム。情報公開法の実効的な行使を市民に可能にし、11万件以上の請求を処理。ジャーナリズム・市民監視の基盤として機能。
  \item \textbf{TheyWorkForYou}——国会議員650名の議会活動を完全に可視化。発言内容のテキスト検索、投票記録の一覧、選挙区ごとの議員活動レポートを提供。月間200万以上のページビュー。
  \item \textbf{FixMyStreet}——地域課題報告プラットフォーム。英国を超えて40か国以上で採用されたモデル。
  \item \textbf{WriteToThem}——選挙区選出議員へのオンライン通信。年間20万件以上のメッセージが送信される。
\end{itemize}

mySocietyのすべてのプラットフォームはオープンソースであり、Alaveteli(WhatDoTheyKnowの基盤)は世界30か国以上で情報公開請求プラットフォームとして採用されている。

\subsubsection{Government Digital Service(GDS)}

英国政府は2011年にGovernment Digital Service(GDS)を設立し、GOV.UKを統一的な政府ウェブサイトとして構築した。GOV.UK Design Systemはオープンソースで公開され、国際的な\govtech{}のベストプラクティスとなっている。GDSの設計原則——「Start with user needs」「Do the hard work to make it simple」——は、\govtech{}の分野で広く参照されている。

\subsubsection{民主主義テクノロジーの展開}

\civictech{}を超えた\politech{}的な取り組みとして、以下のプロジェクトが注目される。
\begin{itemize}[nosep]
  \item \textbf{Democracy Club}——選挙情報のオープンデータ化を推進するボランティア団体。候補者情報・投票所情報をAPI経由で提供。
  \item \textbf{Full Fact AI}——AIを活用した自動ファクトチェックシステムの開発。政治家の発言の事実検証を半自動化。
  \item \textbf{Nesta COLDIGIT}——Nesta(イノベーション財団)による集合知と民主主義のデジタル化に関する研究プログラム。
  \item \textbf{UK e-petitions}——2006年にDowning Street(首相官邸)サイトで開始されたオンライン請願制度。10万筆以上の署名を集めた請願は議会での討議が義務づけられる。2015年に議会ウェブサイトに移管。
\end{itemize}

\subsubsection{6軸評価}

\begin{description}[style=nextline,leftmargin=3em]
  \item[非党派性: 4/5] mySocietyは明示的に非党派であり、TheyWorkForYouは全政党の議員を等しく可視化する。e-petitionsは政府公式だが超党派的に運営されている。
  \item[非企業性: 4/5] mySocietyは非営利団体として運営され、企業利益からの独立性が高い。ただし、助成金への依存度が高く、Nesta等の財団からの資金に依存する。
  \item[オープンソース度: 5/5] mySocietyの全プラットフォーム、GOV.UK Design Systemがオープンソース。英国は\govtech{}/\civictech{}のオープンソース化において世界をリードしている。
  \item[制度的接合性: 3/5] e-petitionsは議会制度に接合されているが、mySocietyのプラットフォームは制度の「外側」から透明性を高めるアプローチであり、意思決定プロセスへの直接的介入は限定的。
  \item[参加の包摂性: 3/5] 英語圏の高いインターネット普及率を背景に、アクセシビリティは比較的高い。ただし、社会経済的格差による参加の偏りは存在する。
  \item[エージェントレディ度: 3/5] TheyWorkForYouはAPIを提供し、構造化された議会データへのプログラマティックアクセスを可能にしている。Democracy Clubも選挙データAPIを提供。ただし、エージェントプロトコルへの対応は未着手。
\end{description}

% ----------------------------------------------------------------------------
\subsection{米国——Code for AmericaとFEC}
\label{subsec:usa}

米国は、\civictech{}の概念を生み出した国であり、市民技術コミュニティの規模と多様性において世界最大である。一方で、企業の政治プロセスへの影響力の大きさが、\politech{}の展開に構造的な制約を課している。

\subsubsection{Code for Americaの展開}

Code for Americaは2009年にJennifer Pahlkaによって設立され、「21世紀にふさわしい政府(Government that works for the people, by the people, in the 21st century)」を掲げてきた\autocite{oecd2025civic}。フェローシッププログラムを通じて、技術者を連邦・州・自治体の行政機関に派遣し、行政サービスのデジタル化を推進してきた。2024年までに6億2,000万世帯以上の行政サービスアクセスを改善したとされる。

Brigadeプログラムは、全米の地域コミュニティにおいて市民技術者の組織化を推進した。ピーク時には80以上のBrigadeが活動し、地域課題のデジタル解決に取り組んだ。ただし、2023年にCode for Americaは組織改革を行い、Brigade プログラムの直接支援を縮小した。

\subsubsection{政治資金透明化エコシステム}

米国における\politech{}的取り組みの最も顕著な領域は、政治資金の透明化である。
\begin{itemize}[nosep]
  \item \textbf{Federal Election Commission(FEC)}——連邦選挙委員会は政治資金データをOpenFEC APIとして公開しており、米国の政治資金は世界で最もデータアクセスが容易である。
  \item \textbf{OpenSecrets(旧Center for Responsive Politics)}——FECデータを分析・可視化し、政治家・企業・ロビイストの資金の流れを市民に分かりやすく提供する非営利団体。
  \item \textbf{GovTrack}——連邦議会の法案追跡・投票記録可視化サイト。2004年設立のパイオニア的存在。
  \item \textbf{OpenStates}——全50州の州議会の法案・投票・議員情報を統合するオープンデータプラットフォーム。
\end{itemize}

\subsubsection{企業影響力の構造的問題}

米国の\politech{}エコシステムは、企業の政治プロセスへの深い関与という構造的課題を抱えている。2010年のCitizens United判決(Citizens United v. Federal Election Commission)は、企業の政治献金を言論の自由として保護し、企業による無制限の政治支出を合法化した。この判決以降、Super PACs(スーパー政治行動委員会)を通じた企業の政治資金投入は急増し、政治プロセスにおける企業の影響力は構造的に拡大している。

さらに、Meta(旧Facebook)、Google、Twitterなどの巨大テクノロジー企業は、市民参加ツールの提供者であると同時に、政治広告の主要プラットフォームでもある。この二重の立場は、\politech{}の非企業性という原則と根本的に矛盾する。2020年・2024年の大統領選挙において、これらプラットフォーム上でのマイクロターゲティング広告が選挙結果に影響を与えたことは広く報告されており\autocite{brookings2024politicization}、技術企業が政治プロセスの中立的な基盤を提供するという想定は、米国においては構造的に成立困難である。

\subsubsection{6軸評価}

\begin{description}[style=nextline,leftmargin=3em]
  \item[非党派性: 3/5] Code for Americaは非党派を掲げるが、Silicon Valleyの文化的バイアスが指摘される。OpenSecretsは超党派的。FECデータは法令に基づく中立的公開。ただし、二大政党制の文脈では「非党派」の実践が構造的に困難。
  \item[非企業性: 2/5] 米国の\civictech{}/\politech{}エコシステムは、企業からの資金提供・技術提供に大きく依存している。Google Civic Information API、Metaの市民参加ツールなど、企業が提供するインフラへの依存度が高い。Citizens United判決以降、政治プロセスにおける企業の影響力は構造的に拡大。
  \item[オープンソース度: 4/5] Code for America、OpenStates、GovTrackはオープンソース。FECデータはオープンデータとして公開。ただし、主要なプラットフォーム(Meta、Google)のアルゴリズムはプロプライエタリ。
  \item[制度的接合性: 3/5] FECによる政治資金データの法定公開、e-Rulemakingによる行政規則制定過程への市民参加は制度化されている。ただし、市民の熟議が立法に直接反映される仕組みは乏しい。
  \item[参加の包摂性: 2/5] 人種・所得・教育水準による参加格差が大きい。デジタルデバイドに加え、投票抑制(voter suppression)が構造的問題として存在。フェロン・ディスフランチャイズメント(重罪者の投票権剥奪)は530万人以上に影響。
  \item[エージェントレディ度: 3/5] OpenFEC API、Congress APIなどの構造化されたAPIが存在し、プログラマティックアクセスは比較的容易。ただし、エージェントプロトコルへの対応やAIエージェントの参入を前提とした設計は未着手。
\end{description}

% ----------------------------------------------------------------------------
\subsection{欧州——DecidimとCONSUL}
\label{subsec:europe}

欧州は、参加型民主主義のデジタルプラットフォームにおいて、世界で最も多様な実験を展開している地域である。特にスペインのDecidimとCONSULは、オープンソースの熟議プラットフォームとして国際的に最も広く採用されている。

\subsubsection{Decidim(バルセロナ)}

Decidimは2015年にバルセロナ市議会によって開発が開始されたオープンソースの参加型民主主義プラットフォームであり、その名称はカタルーニャ語で「我々は決める(We decide)」を意味する。

Decidimの技術的特徴は以下の通りである。
\begin{itemize}[nosep]
  \item \textbf{モジュラー・アーキテクチャ}——Ruby on Railsで構築され、提案(Proposals)、熟議(Debates)、会議(Meetings)、参加型予算(Participatory Budgets)、調査(Surveys)、法案追跡(Accountability)などのコンポーネントを自由に組み合わせ可能。
  \item \textbf{メタ参加(Metadecidim)}——Decidimの開発ロードマップ自体がDecidimプラットフォーム上で決定されるという自己再帰的構造。プラットフォームの設計に市民が参加する「プラットフォームのための参加」。
  \item \textbf{社会契約}——Decidimコミュニティは「Social Contract」を掲げ、自由ソフトウェア、透明性、協働、データ主権を原則として明文化している。
  \item \textbf{国際展開}——2025年現在、世界500以上の機関(自治体・大学・NGO・政党)で採用されている。
\end{itemize}

バルセロナ市における実装では、2016年の市民参加型戦略計画策定において4万人以上の市民が参加し、7,000件以上の政策提案が集約された。参加型予算では年間数百万ユーロ規模の予算配分が市民投票によって決定されている。

\subsubsection{CONSUL Democracy(マドリード)}

CONSULはマドリード市議会が2015年に開発を開始したオープンソースの市民参加プラットフォームであり、Decidimとは独立に発展してきた。Ruby on Railsで構築され、提案・討議・投票・参加型予算の機能を提供する。2025年現在、35か国以上の135以上の機関で採用されている。

CONSULの特徴はDecidimに比してインストール・運用が容易であり、小規模自治体での採用が多い点にある。ただし、Decidimのようなモジュラー・アーキテクチャの柔軟性やメタ参加の仕組みは持たず、カスタマイズの余地はやや限定的である。

\subsubsection{LiquidFeedback(ドイツ)}

LiquidFeedback は2009年にドイツ海賊党(Piratenpartei Deutschland)の党内意思決定プラットフォームとして開発され、液体民主主義(Liquid Democracy)の概念を実装した先駆的プラットフォームである。液体民主主義とは、有権者が各政策課題について直接投票するか、信頼する代理人に委任するかを選択でき、さらに委任を連鎖的に行うことも可能な柔軟な代表制の仕組みである。

LiquidFeedbackはドイツ海賊党の急速な台頭(2011--2012年)において中心的な役割を果たしたが、その後の党の衰退とともに利用は減少した。しかし、液体民主主義の概念は、その後のDemocracy.Earthなどのプロジェクトに継承されている。LiquidFeedbackの経験は、\politech{}プラットフォームが特定の政党と結びつくことの脆弱性を示す教訓でもある。

\subsubsection{フランス——気候市民会議}

フランスの Convention Citoyenne pour le Climat(気候のための市民会議、2019--2020年)は、抽選で選ばれた150人の市民が気候変動対策を議論し、149の提案を政府に提出した熟議型民主主義の大規模実験である。デジタルプラットフォームを主要な基盤としたものではないが、市民抽選(sortition)による代表性の確保と、長期間にわたる深い熟議のモデルとして、\politech{}の「参加の包摂性」と「制度的接合性」に関して重要な示唆を提供する。

しかし、149の提案のうちMacron大統領が実際に採用したのは一部にとどまり、「市民の熟議→政策反映」の経路の脆弱性を露呈した。制度的接合性が不十分な場合、市民の熟議が政策に反映されず、参加者の幻滅を招くという教訓を残している。

\subsubsection{アイスランド——憲法クラウドソーシング}

アイスランドは2010--2011年に、金融危機後の憲法改正プロセスにおいて、市民のクラウドソーシングによる憲法草案の作成を試みた。950人の無作為抽出市民による国民フォーラム、25人の選出された市民による憲法審議会、SNSを通じた市民コメントの収集が組み合わされ、参加型の憲法草案が作成された。

2012年の国民投票で67\%の賛成を得たにもかかわらず、議会(Althingi)は憲法草案の承認を見送った。この事例は、市民の直接参加による正統性と議会制民主主義の制度的権限の間の緊張関係を如実に示している。

\subsubsection{EU Digital Services Act}

EUは2022年にDigital Services Act(DSA)を成立させ、巨大プラットフォーム企業に対するアルゴリズム透明性の義務、虚偽情報対策、市民の権利保護を規定した。DAAは直接的な\politech{}立法ではないが、プラットフォームの透明性・説明責任という\politech{}の前提条件を法的に整備するものとして重要である。

\subsubsection{6軸評価}

\begin{description}[style=nextline,leftmargin=3em]
  \item[非党派性: 4/5] Decidim・CONSULはいずれも超党派的に設計されている。Decidimの社会契約は明示的に非党派性を掲げる。ただし、LiquidFeedbackのように政党との結びつきが脆弱性となった事例もある。
  \item[非企業性: 5/5] Decidim・CONSULはいずれも公的機関主導で開発され、企業利益からの独立性が高い。Decidimの社会契約はデータ主権を明文化。EUのDSAは企業の政治的影響力に対する法的制約を提供。
  \item[オープンソース度: 5/5] Decidim・CONSULはAGPLライセンスの下で完全にオープンソース。Decidimのメタ参加はオープンソースガバナンスの模範的事例。
  \item[制度的接合性: 3/5] バルセロナのDecidimは市の意思決定に制度的に組み込まれている。フランスの気候市民会議は制度的接合の不十分さを露呈。アイスランドの事例は議会による拒否という限界を示した。制度的接合性にばらつきが大きい。
  \item[参加の包摂性: 4/5] フランスの市民抽選は包摂性の確保に最も成功した手法。Decidimは多言語対応・アクセシビリティに配慮。ただし、デジタルプラットフォームへの参加は都市部・高学歴層に偏る傾向。
  \item[エージェントレディ度: 2/5] DecidimはREST APIを提供しているが、エージェントプロトコルへの対応は未着手。AIエージェントの参入を前提とした設計は行われていない。
\end{description}

% ----------------------------------------------------------------------------
\subsection{日本——Code for Japanとチームみらい}
\label{subsec:japan}

日本の\politech{}エコシステムは、豊かな\civictech{}コミュニティの蓄積を有しながらも、制度的接合性とエージェントレディ度において大きな課題を抱えている。

\subsubsection{Code for Japanエコシステム}

Code for Japanは2013年に関治之によって設立された日本最大の\civictech{}コミュニティである。Jennifer PahlkaのCode for AmericaにおけるTEDトーク(2012年)に触発され、日本版の市民技術コミュニティとして発足した。2011年の東日本大震災時に市民が自発的に構築したSinsai.info(被災情報集約サイト)が直接の起源であり、災害対応における市民技術の有効性を実証した経験が組織化の基盤となった。

2025年現在、Code for Japanは80以上の地域コミュニティ(Brigade)を組織し、以下の活動を展開している。
\begin{itemize}[nosep]
  \item \textbf{行政との協働事業}——自治体との協働によるデジタルサービスの開発・改善。行政コンサルティングが主要な収益源。
  \item \textbf{Decidim日本展開}——Decidimの日本語化と国内自治体への導入支援。加古川市、兵庫県など30以上の自治体が採用。
  \item \textbf{Social Technology Officer(STO)}——自治体にデジタル推進の専門人材を派遣するプログラム。
  \item \textbf{ハッカソン・イベント}——定期的なハッカソン、シビックテックフォーラムの開催。
\end{itemize}

\subsubsection{チームみらい/デジタル民主主義2030}

2024年に設立された「チームみらい」は、安野たかひろ(2024年東京都知事選候補者)を中心とする政治団体であり、「デジタル民主主義2030(DD2030)」構想を掲げている。

チームみらいの技術的取り組みは以下の通りである。
\begin{itemize}[nosep]
  \item \textbf{広聴AI}——市民の声をAIで分析・要約し、政策立案に反映させるシステム。LLMを活用した意見クラスタリングと要約生成。
  \item \textbf{Polimoney}——政治資金の可視化プラットフォーム。
  \item \textbf{Pol.is活用}——政策課題に関する市民意見のPol.isによる集約実験。
\end{itemize}

チームみらいは、日本において\politech{}を明示的に掲げる数少ない団体であるが、特定の政治的リーダーとの結びつきが強く、非党派性の観点では課題がある。

\subsubsection{制度的環境}

日本の制度的環境は、\politech{}の展開に対して以下の特徴を有する。
\begin{itemize}[nosep]
  \item \textbf{デジタル庁}——2021年設立。行政のデジタル化を推進するが、\politech{}は所管外。
  \item \textbf{政治資金規正法}——2024年改正後も、オンライン公開の義務化は全政治団体の約5\%にとどまる\autocite{brookings2024politicization}。
  \item \textbf{国会議事録API}——国立国会図書館がkokkai.ndl.go.jpでAPIを提供しているが、利活用は限定的。
  \item \textbf{文化的要因}——「お上意識」(行政への従順な態度)、根回し(非公式な事前合意形成)の文化は、オープンな熟議とは異なる意思決定様式を支持する。
\end{itemize}

\subsubsection{6軸評価}

\begin{description}[style=nextline,leftmargin=3em]
  \item[非党派性: 3/5] Code for Japanは非党派を掲げ実践しているが、チームみらいは特定の政治的リーダーとの結びつきがある。日本の\civictech{}コミュニティは概ね非党派的だが、政策提言を行う際に党派的に受け取られるリスクがある。
  \item[非企業性: 3/5] Code for Japanは非営利だが、行政コンサルティング収入への依存度が高い。チームみらいのPolimoneyは独立性を掲げるが、資金構造の透明性は限定的。日本の\civictech{}は全般的にボランティアベースだが、持続可能性に課題。
  \item[オープンソース度: 3/5] Code for Japanのプロジェクトはオープンソースが多いが、行政との協働事業ではプロプライエタリなコードも含まれる。Decidim日本版はオープンソース。国会議事録APIは公開されているが、政治資金データのオープンデータ化は大幅に遅れている。
  \item[制度的接合性: 2/5] Decidimを採用した自治体では一定の制度的接合があるが、国政レベルでの\politech{}の制度化は皆無。国会・中央省庁と\civictech{}コミュニティの制度的連携は極めて限定的。
  \item[参加の包摂性: 2/5] 80以上のBrigadeが全国に展開しているが、参加者は技術者コミュニティに偏る。高齢者・デジタルリテラシーの低い層の参加は限定的。投票率の低さ(衆院選50\%台)に示されるように、そもそも政治参加の意欲が低い文化的背景がある。
  \item[エージェントレディ度: 1/5] 国会議事録APIは存在するが構造化度が低く、政治資金データのAPIは存在しない。選挙データの機械可読性も低い。エージェントプロトコルへの対応は皆無。日本の政治データインフラは、AIエージェントの活用を前提とした設計から最も遠い位置にある。
\end{description}

% ----------------------------------------------------------------------------
\subsection{5地域の総合比較}
\label{subsec:comprehensive-comparison}

以上の分析を総合し、5地域の6軸評価を表\ref{tab:six-axis-comparison}にまとめる。

\begin{table}[htbp]
\centering
\caption{5地域の6軸比較(各軸0--5点)}
\label{tab:six-axis-comparison}
\small
\newcolumntype{C}{>{\centering\arraybackslash}X}
\begin{tabularx}{\textwidth}{lCCCCC}
\toprule
\textbf{比較軸} & \textbf{台湾} & \textbf{英国} & \textbf{米国} & \textbf{欧州} & \textbf{日本} \\
\midrule
非党派性 & 4 & 4 & 3 & 4 & 3 \\
非企業性 & 5 & 4 & 2 & 5 & 3 \\
オープンソース度 & 5 & 5 & 4 & 5 & 3 \\
制度的接合性 & 4 & 3 & 3 & 3 & 2 \\
参加の包摂性 & 3 & 3 & 2 & 4 & 2 \\
エージェントレディ度 & 2 & 3 & 3 & 2 & 1 \\
\midrule
\textbf{合計} & \textbf{23} & \textbf{22} & \textbf{17} & \textbf{23} & \textbf{14} \\
\bottomrule
\end{tabularx}
\end{table}

この比較から以下の知見が導出される。

\paragraph{知見1: 台湾と欧州が最も包括的な\politech{}エコシステムを有する。}
台湾はg0v/vTaiwanモデルによる市民主導のボトムアップ型\politech{}において、欧州はDecidim/CONSULによる制度的プラットフォーム型\politech{}において、それぞれ先進的な地位を占めている。両地域に共通するのは、非企業性とオープンソース度の高さである。

\paragraph{知見2: 米国は企業影響力が\politech{}の構造的制約となっている。}
米国は\civictech{}の発祥地であり、政治資金データの公開度では世界最高水準にあるが、企業の政治プロセスへの深い関与が\politech{}の非企業性・非党派性を構造的に制約している。

\paragraph{知見3: エージェントレディ度はすべての地域で低い。}
5地域すべてにおいて、AIエージェントの参入を前提とした設計は初期段階にある。これは、エージェントレディ度が\politech{}の比較フレームワークに本格的に組み込まれていないことを反映している。本論文が提案する6軸フレームワークの中で、エージェントレディ度は最も新しい——そして今後最も重要になりうる——分析軸である。

\paragraph{知見4: 日本は6軸すべてにおいて改善の余地がある。}
日本は80以上のCode for Japanコミュニティという豊かな\civictech{}の蓄積を有しながらも、制度的接合性とエージェントレディ度において最も低い評価となった。特にエージェントレディ度の1/5は、政治データの機械可読性とAPI基盤の整備が大幅に遅れていることを反映している。この課題に対する応答として、第\ref{sec:japan-case-study}節でOJPPの設計を検討する。
