% ============================================================================
% Section 1: Introduction
% ============================================================================

\section{序論——なぜ政党にも企業にもよらない政治技術が必要か}
\label{sec:introduction}

21世紀の民主主義は、かつてない信頼の危機に直面している。Eurobarometerの2024年調査によれば、欧州市民の政党に対する信頼率はわずか17\%にとどまり\autocite{eurobarometer2024}、Pew Research Centerの2023年調査では米国連邦政府を信頼する市民は20\%を下回る\autocite{pew2023trust}。この信頼の崩壊は、単なる一時的な現象ではなく、半世紀にわたる構造的趨勢である。

一方で、デジタル技術を用いた民主主義の刷新は、世界各地で実験的に進行している。台湾のvTaiwanは、分極化を乗り越える集団的意見形成の基盤として国際的に注目を集め\autocite{tang2024plurality}、スペイン・バルセロナのDecidimは市民参加型予算編成の基盤として40か国以上に展開されている。ドイツのLiquidFeedbackは液体民主主義(Liquid Democracy)の概念実装として、委任の連鎖による柔軟な代表制を模索してきた。注目すべきは、これらの試みがいずれも政党主導ではなく、市民社会・研究者・技術者の連合体によって推進されてきたという事実である。

本論文は、この事実に理論的根拠を与える。すなわち、政治プロセスのデジタル化は、構造的に、政党にも企業にも委ねるべきではない——非党派的(non-partisan)、非企業的(non-corporate)、オープンソース(open-source)、かつAIエージェントの参入を前提とした(agent-ready)設計こそが、民主主義の持続可能性を担保する構造的要件であることを論証する。


% ----------------------------------------------------------------------------
\subsection{問題の所在}
\label{subsec:problem-statement}

政治のデジタル化をめぐる議論は、これまで二つの軸に沿って展開されてきた。第一は「行政のデジタル化」(\govtech{})であり、電子政府・行政手続のオンライン化・データに基づく政策立案(Evidence-Based Policy Making; EBPM)を指す。エストニアのe-Residencyに代表されるように、\govtech{}は既に多くの国で制度化が進んでいる。第二は「市民参加の技術」(\civictech{})であり、市民が行政や政策過程に関与するためのツール群を指す。Code for America・Code for Japanに代表される市民技術コミュニティがこの領域を牽引してきた\autocite{oecd2025civic}。

しかし、この二分法には重大な盲点がある。「何を、誰が、どのように決めるか」という政治の意思決定プロセスそのもの——すなわち、利益集約・政策形成・合意形成・代表選出・政策監視といった民主主義の核心的機能——のデジタル化は、\govtech{}にも\civictech{}にも十分には包摂されない。本論文はこの第三の領域を「\politech{}(政治技術)」と定義し、その設計原則を導出する。

\politech{}の設計において、決定的に重要な問いが二つある。第一に、なぜ政党が政治のデジタル化を主導すべきではないのか。第二に、なぜ企業が政治のデジタル化を担うべきではないのか。本論文の序論では、この二つの問いに対して構造的な回答を与える。

デジタル民主主義の先行研究は、この二重の問いに対して示唆的な知見を提供している。\textcite{fishkin2011people}は討議型世論調査(Deliberative Polling)を通じて、情報提供と熟議が市民の選好を変容させることを示し、政党による事前のフレーミングが熟議の質を歪めるリスクを指摘した。\textcite{landemore2020open}は「開かれた民主主義」の構想において、選挙を経ない市民の直接的な政策関与を理論化し、政党の媒介機能の限界を論じた。\textcite{dryzek2010foundations}は熟議民主主義の基盤理論として、真正な熟議が権力関係から解放された空間で行われるべきことを論じ、政党や企業のような利害関係者が熟議空間を管理することの問題性を理論的に導出した。

これらの理論的知見に加え、近年のAI技術の急速な発展は、問題をさらに複雑化させている。Stanford HAI AI Indexの2025年報告は、大規模言語モデル(LLM)が体系的な党派的偏向を内包していることを明らかにし\autocite{stanfordhai2025aiindex}、AI技術を政治プロセスに統合する際の中立性確保が喫緊の課題であることを示した\autocite{stanfordhai2025neutrality}。政治のデジタル化が単なる行政効率の問題ではなく、民主主義の正統性に関わる根本的な設計問題であることが、いまや明白である。


% ----------------------------------------------------------------------------
\subsection{政党が政治デジタル化を主導することの構造的問題}
\label{subsec:partisan-problems}

政党が政治のデジタル化を主導することには、以下の三つの構造的問題がある。

\subsubsection{党派的偏向の埋め込み問題}

政治プラットフォームの設計は、不可避的に価値選択を伴う。議題設定のアルゴリズム、意見集約の方法、可視化のデザイン、そしてモデレーションの基準——これらすべてが、設計者の価値判断を反映する。政党がプラットフォームを設計する場合、その党派的利益が設計に埋め込まれることは構造的に不可避である。

この問題は、AI技術の導入によってさらに深刻化する。Stanford HAIの2025年報告\autocite{stanfordhai2025neutrality}は、主要なLLMが体系的な政治的偏向を示すことを実証している。GPT-4、Claude、Geminiなどのモデルは、左派リベラル寄りの回答傾向を示し、この偏向はプロンプトエンジニアリングでは完全には矯正できない。ワシントン大学の2025年研究は、AIチャットボットとの対話が利用者の政治的見解を有意に変容させることを示しており、LLMを用いた政治プラットフォームにおける党派的偏向の埋め込みは、単なる情報の歪曲にとどまらず、市民の選好そのものを操作するリスクを内包する。

政党がプラットフォームの設計主体となる場合、使用するAIモデルの選択、ファインチューニングのデータセット、出力のフィルタリング基準のすべてが、党派的利益に影響される。この構造的偏向は、プラットフォームの中立性を根本から損なう。

\subsubsection{利益相反問題}

政治のデジタル化の核心的目的の一つは、政治プロセスの透明性の向上にある。しかし、透明性の向上は多くの場合、現職者(incumbent)の利益に反する。政治資金の流れの可視化、投票記録の体系的公開、政策決定過程の文書化——これらはすべて、現在権力を握る政党にとって、自らの行動を監視される仕組みの構築を意味する。

日本の事例はこの問題を端的に示している。2024年に改正された政治資金規正法は、旧来の政治資金パーティーの収支報告義務の強化を掲げたが、その適用対象は全政治資金管理団体のわずか約5\%にとどまった\autocite{brookings2024politicization}。デジタル技術によって政治資金の完全な透明化が技術的には可能であるにもかかわらず、それが実現しないのは、技術的制約ではなく、現職政党の構造的な利益相反に起因する。

すなわち、政治の透明性を高めるツールの設計・運営を、その透明性によって最も不利益を被る主体に委ねることは、論理的矛盾である。

\subsubsection{プラットフォーム持続性問題}

政党が運営するデジタルプラットフォームは、政権交代・党内再編・選挙敗北といった政治的変動に対して脆弱である。政党は本質的に選挙サイクルに規定された組織であり、その技術インフラもまた政治的命運と一体化する。

政党が構築したプラットフォームは、その政党が下野すれば運営予算を失い、党が解散すれば消滅する。蓄積された市民の意見データ、熟議の記録、政策提案のアーカイブは、政党のインフラに依存する限り、政治的変動とともに散逸する。\textcite{margetts2016turbulence}が指摘するように、デジタル政治は本質的に「乱流(turbulence)」を伴うものであり、政党のように政治的変動に直接さらされる組織に、長期的な民主主義インフラの維持を委ねることは適切ではない。

民主主義のインフラストラクチャは、特定の政党の栄枯盛衰から独立して持続する必要がある。これは道路や橋梁といった物理的インフラが特定の政権から独立して維持されるべきであるのと同様の論理構造である。


% ----------------------------------------------------------------------------
\subsection{企業が政治デジタル化を担うことの構造的問題}
\label{subsec:corporate-problems}

政党と同様に、企業もまた政治のデジタル化を担う主体としては構造的な限界を抱えている。以下の四つの問題を指摘する。

\subsubsection{利潤動機との相克}

営利企業の第一義的な目的は株主利益の最大化であり、この目的と民主主義的価値の追求は構造的に緊張関係にある。Facebook(現Meta)社の内部文書が2021年に流出した「Facebookファイル」は、この緊張関係を最も端的に示す事例である。同社は、エンゲージメント最大化アルゴリズムが政治的分極化と偽情報の拡散を促進することを内部で認識していたにもかかわらず、広告収益への影響を懸念してアルゴリズムの修正を見送った。

政治プラットフォームの設計目的が「利用者のエンゲージメント最大化」や「データ収集の最大化」であるとき、それは必然的に熟議の質を犠牲にする。感情的な発言は冷静な議論よりも多くのエンゲージメントを生み、分極的な論点は合意形成よりも多くのアクセスを集める。\textcite{margetts2016turbulence}が明らかにしたように、デジタル空間における政治参加は「乱流」の特性を持ち、企業のプラットフォーム設計がこの乱流を増幅する方向に作用する構造的誘因が存在する。

\subsubsection{プロプライエタリ設計による検証不可能性}

企業が政治プラットフォームを運営する場合、その中核アルゴリズムは営業秘密(trade secret)として非公開となる。意見集約のアルゴリズム、コンテンツ推薦のロジック、モデレーションの基準、AIモデルの学習データ——これらはすべて企業の知的財産として保護され、外部からの独立した検証が不可能となる。

民主主義プロセスにおいて、意思決定の仕組みが外部から検証できないことは、正統性(legitimacy)の根本的な毀損を意味する。市民が自らの意見がどのように集約され、どのように政策に反映されるかを検証できないプラットフォームは、たとえその出力が公正であったとしても、民主主義的正統性を欠く。ブラックボックスの中で行われる意思決定は、それが正しいか否かにかかわらず、民主主義的ではない。

\subsubsection{企業利益の設計への埋め込み}

企業が運営するプラットフォームは、その企業の事業戦略に規定される。データ収集の範囲と利用目的、API(Application Programming Interface)のアクセス条件、料金体系、サードパーティ連携の可否——これらの設計判断は、企業の収益モデルに従属する。

たとえば、企業が政治プラットフォームの利用データを広告ターゲティングに流用する可能性、あるいは特定のAPI利用者に高額な料金を設定することで事実上のアクセス制限を課す可能性は、構造的に排除できない。これらの設計判断は、政治参加の平等性(political equality)を損なうリスクを内包する。

International IDEAの2024年報告\autocite{idea2024digital}は、デジタル民主主義の基盤となるプラットフォームが特定の企業に依存するリスクを指摘し、公共的なデジタルインフラの必要性を論じている。政治参加のインフラストラクチャが特定企業の経営判断に左右される状況は、民主主義の基盤として脆弱である。

\subsubsection{サービス継続性リスク}

企業が提供するデジタルサービスは、経営判断によっていつでも終了されうる。Google+(2019年終了)、Vine(2017年終了)、Yahoo! Answers(2021年終了)などの事例は、大企業が運営するプラットフォームであっても、事業上の判断によってサービスが打ち切られることを示している。

政治プラットフォームがこのようなサービス終了リスクにさらされることは、民主主義インフラとして致命的である。市民参加の記録、熟議のアーカイブ、政策提案のデータベースが、一企業の経営判断によって消失しうる状況は、許容されるべきではない。

さらに、企業の買収・合併・経営方針の転換もリスク要因となる。Twitterが2022年にElon Muskに買収された後、プラットフォームのポリシーが大幅に変更された事例は、政治的言論の基盤が一個人の意思決定に左右されうることを如実に示している。

これら四つの問題は、いずれも企業の善意や個別の経営判断によって解決される性質のものではなく、営利企業が政治インフラを担うことに内在する構造的問題である。


% ----------------------------------------------------------------------------
\subsection{政党の機能の再検討}
\label{subsec:party-functions}

ここまでの議論は、政党や企業が政治のデジタル化を「主導」することの問題を指摘したものであり、政党の存在意義そのものを否定するものではない。しかし、\politech{}の射程を明確化するためには、政党が伝統的に担ってきた機能を分解し、どの機能が政党に固有であり、どの機能が技術的に代替可能であるかを検討する必要がある。

政治学の標準的な教科書が列挙する政党の機能は、以下のように整理される。

\begin{enumerate}[label=(\arabic*)]
  \item \textbf{利益集約機能}(Interest Aggregation)——多様な市民の選好を集約し、政策パッケージとしてまとめる機能。
  \item \textbf{政策形成機能}(Policy Formulation)——集約された利益をもとに具体的な政策案を策定する機能。
  \item \textbf{候補者選出機能}(Candidate Selection)——選挙に立候補する人材を発掘・選出・支援する機能。
  \item \textbf{選挙組織化機能}(Campaign Organization)——選挙運動を組織し、有権者の動員を図る機能。
  \item \textbf{統治機能}(Governance)——政権を担い、立法・行政を運営する機能。
  \item \textbf{野党機能}(Opposition)——政権を監視し、批判と代替案を提示する機能。
  \item \textbf{政治教育機能}(Civic Education)——市民の政治的リテラシーを向上させる機能。
\end{enumerate}

これらの機能のうち、政党に固有であり技術的に代替が困難な機能は、候補者選出機能(3)、選挙組織化機能(4)、および統治機能(5)である。これらの機能は、人間の判断、個人的信頼関係、組織的動員力、そして憲法上の権限行使を本質的に含んでおり、技術的代替の対象とはなりにくい。

一方、以下の機能は、技術的に代替可能であるか、少なくとも技術によって大幅に拡張・補完されうる。

\paragraph{利益集約機能の技術的代替}
Pol.isに代表されるブロードリスニング・プラットフォームは、大規模な意見集約を政党の媒介なしに実現する。Pol.isは、参加者の投票パターンを主成分分析によって可視化し、意見の分布と合意点を自動的に抽出する。台湾のvTaiwanは、この技術を用いてUberX規制やオンラインアルコール販売規制などの政策課題について、政党を介さない利益集約を成功させた\autocite{tang2024plurality}。

\paragraph{政策形成機能の技術的代替}
Decidimは、市民が直接的に政策提案を行い、熟議を経て修正・統合するプロセスを、デジタルプラットフォーム上で実現している。バルセロナ市の戦略計画策定において、Decidimは4万人以上の市民参加を組織し、7,000件以上の政策提案を集約した。この過程は、政党の政策形成機能を部分的に代替するものである。

\paragraph{野党機能の技術的代替}
オープンデータと\civictech{}の組み合わせは、政権監視機能を市民社会に拡張する。議会議事録の自動分析、政治資金の可視化、政策効果の独立評価——これらはすべて、技術を用いて野党機能の一部を市民社会に分散させることを可能にする。\textcite{oecd2025civic}は、OECDの報告において\civictech{}が政府の説明責任(accountability)を強化する効果を実証的に示している。

\paragraph{政治教育機能の技術的代替}
AIを活用した政治教育は、市民が政策の背景・影響・トレードオフを理解するための情報提供を、大規模かつ個別化された形で実現しうる。LLMを用いた対話型の政策解説、シミュレーションに基づく政策影響の可視化、多言語対応の情報提供——これらは、政党が伝統的に担ってきた政治教育機能を、非党派的に代替する可能性を示している。

以上の分析から、本論文の中心的命題が導かれる。すなわち、政党が伝統的に担ってきた機能の相当部分は、非党派的・非企業的な技術基盤と市民社会の連合によって代替可能であり、\politech{}はこの代替を体系的に実現するための設計原則と技術基盤を提供するものである。

ただし、本論文はこの代替を、政党の廃止として主張するものではない。政党は選挙制度と統治機構に不可分に結びついた制度的存在であり、候補者選出・選挙組織・統治の機能については、当面の間、政党に代わる制度的仕組みは存在しない。本論文が提案するのは、政党の機能のうち技術的に代替可能な部分を、非党派的な公共インフラとして分離・独立させることである。これは、政党を弱体化させるのではなく、政党が本来集中すべき機能——候補者の発掘・統治の遂行——に資源を集中させることを可能にする、補完的関係の構築である。


% ----------------------------------------------------------------------------
\subsection{本論文の目的と構成}
\label{subsec:purpose-and-structure}

以上の問題意識に基づき、本論文の目的を以下のように定める。

\begin{quote}
本論文は、政治のデジタル化における非党派的・非企業的・オープンソース・エージェントレディな設計の構造的優位性とその限界を、国際比較分析を通じて明らかにし、\politech{}の理論的基盤と設計原則を導出することを目的とする。
\end{quote}

この目的を達成するために、本論文は以下の四つの研究課題(Research Questions)を設定する。

\begin{description}[style=nextline,leftmargin=2.5em,labelindent=0em]
  \item[\textbf{RQ1}:] \govtech{}・\civictech{}・\politech{}はいかに概念的に区別されるべきか。三概念の理論的境界と相互関係を明確化し、\politech{}の独自の射程を定義する。
  \item[\textbf{RQ2}:] 非党派的・非企業的・オープンソースのアプローチは、政治のデジタル化においていかなる構造的優位性を持ち、またいかなる限界を抱えるか。
  \item[\textbf{RQ3}:] AIエージェントの政治プロセスへの参入を前提とした「エージェントレディ」な政治インフラは、いかに設計されるべきか。エージェントの認証・権限管理・透明性確保のための技術的・制度的要件を明らかにする。
  \item[\textbf{RQ4}:] 政治のデジタル化を既存の政治制度(議会・地方自治体・選挙制度)とどのように接続すべきか。設計原則と制度設計の相互作用を解明する。
\end{description}

\paragraph{学術的貢献}
本論文は、以下の五つの学術的貢献を行う。

\begin{enumerate}[label=\textbf{C\arabic*}:]
  \item \textbf{\politech{}の概念構築}——\govtech{}と\civictech{}の間隙にある政治の意思決定プロセスのデジタル化を、独自の概念として定義・理論化する。Arrowの不可能性定理、Habermasの討議倫理、計算論的社会選択理論(Computational Social Choice)を統合し、\politech{}の理論的基盤を構築する。
  \item \textbf{ブロードリスニング・プラットフォームの包括的サーベイ}——Polis、vTaiwan、Decidim、Talk to the City、広聴AIなど、世界各地で展開されるブロードリスニング技術の体系的な比較分析を行う。
  \item \textbf{AI$\times$民主主義研究の体系的位置づけ}——Habermas Machine\autocite{fishkin2011people}、Collective Constitutional AI、Generative Social Choiceなど、最先端のAI×民主主義研究を\politech{}の文脈に位置づける。
  \item \textbf{6軸比較フレームワークによる国際比較}——台湾・英国・米国・欧州・日本の5地域における政治デジタル化の事例を、技術・制度・市民参加・透明性・持続可能性・エージェントレディ性の6軸で比較分析する。
  \item \textbf{エージェントレディ設計の原則導出}——AIエージェントの政治プロセスへの参入を前提とした設計原則を導出し、Open Japan PoliTech Platform(OJPP)の実装を通じてその実現可能性を示す。
\end{enumerate}

\paragraph{論文構成}
本論文は以下の9節から構成される。第\ref{sec:introduction}節(本節)では問題の所在を提示し、政党・企業が政治デジタル化を主導することの構造的問題を論じた。第2節では、Arrowの不可能性定理からHabermasの討議倫理、計算論的社会選択理論に至る民主主義理論の系譜を概観し、\politech{}の理論的基盤を構築する。第3節では、ブロードリスニング・プラットフォームおよびAI×民主主義研究の包括的文献調査を行う。第4節では、\govtech{}・\civictech{}・\politech{}の三概念を比較し、\politech{}の独自の射程を定義する。第5節では、台湾・英国・米国・欧州・日本の5地域における国際比較分析を行う。第6節では、日本のケーススタディとして、広聴AIの導入事例とOJPPの設計を検討する。第7節では、エージェントレディな政治インフラの設計原則を導出する。第8節では、本論文の知見を総合的に考察し、理論的・実践的含意を論じる。第9節では結論を述べ、今後の研究課題を提示する。
